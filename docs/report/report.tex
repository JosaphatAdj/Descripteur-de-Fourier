\documentclass[a4paper,12pt]{report}
\usepackage[utf8]{inputenc}
\usepackage[T1]{fontenc}
\usepackage[french]{babel}
\usepackage{graphicx}
\usepackage{hyperref}
\usepackage{listings}
\usepackage{xcolor}
\usepackage{geometry}
\geometry{hmargin=2.5cm,vmargin=2.5cm}

% Configuration des hyperliens
\hypersetup{
    colorlinks=true,
    linkcolor=blue,
    filecolor=magenta,      
    urlcolor=cyan,
}

% Configuration des listings de code
\definecolor{codegreen}{rgb}{0,0.6,0}
\definecolor{codegray}{rgb}{0.5,0.5,0.5}
\definecolor{codepurple}{rgb}{0.58,0,0.82}
\definecolor{backcolour}{rgb}{0.95,0.95,0.92}

\lstdefinestyle{mystyle}{
    backgroundcolor=\color{backcolour},   
    commentstyle=\color{codegreen},
    keywordstyle=\color{magenta},
    numberstyle=\tiny\color{codegray},
    stringstyle=\color{codepurple},
    basicstyle=\ttfamily\footnotesize,
    breakatwhitespace=false,         
    breaklines=true,                 
    captionpos=b,                    
    keepspaces=true,                 
    numbers=left,                    
    numbersep=5pt,                  
    showspaces=false,                
    showstringspaces=false,
    showtabs=false,                  
    tabsize=2
}

\lstset{style=mystyle}

\title{\textbf{Optimisation du Calcul des Descripteurs de Fourier avec OpenBLAS}}
\author{Projet OpenBLAS}
\date{\today}

\begin{document}

\maketitle
\tableofcontents

\chapter{Introduction}
\section{Contexte : Analyse de formes industrielle}
L'analyse de formes est une composante critique dans l'industrie manufacturière, notamment pour le contrôle qualité automatisé. La détection de défauts sur des pièces de fonderie nécessite des descripteurs robustes, invariants aux transformations géométriques (rotation, échelle, translation). Les descripteurs de Fourier répondent à ce besoin, mais leur calcul peut être coûteux sur de grands volumes de données.

\section{Objectif}
Ce projet vise à évaluer l'apport de la bibliothèque d'algèbre linéaire optimisée \textbf{OpenBLAS} pour accélérer le calcul des descripteurs de Fourier. Nous comparons une implémentation naïve en C avec une version vectorisée utilisant les routines BLAS Level 1 (\texttt{cblas\_ddot}, \texttt{cblas\_dnrm2}).

\chapter{Fondements Mathématiques}
\section{Descripteurs de Fourier}
Un contour fermé peut être représenté comme une fonction périodique de coordonnées complexes :
\[ z(k) = x(k) + i y(k), \quad k=0,\dots,N-1 \]

La Transformée de Fourier Discrète (TFD) donne les coefficients :
\[ c_n = \frac{1}{N} \sum_{k=0}^{N-1} z(k) e^{-i 2\pi n k / N} \]

\section{Normalisation}
Pour rendre les descripteurs invariants :
\begin{itemize}
    \item \textbf{Translation} : On ignore le coefficient $c_0$.
    \item \textbf{Rotation} : On utilise uniquement le module $|c_n|$.
    \item \textbf{Échelle} : On normalise par la norme L1 des magnitudes : $FD_n = \frac{|c_n|}{\sum |c_k|}$.
\end{itemize}

\chapter{Architecture OpenBLAS}
OpenBLAS est une implémentation open-source optimisée des API BLAS (Basic Linear Algebra Subprograms). Elle tire parti :
\begin{itemize}
    \item Du parallélisme multi-coeurs.
    \item Des instructions vectorielles SIMD (AVX2, AVX-512).
    \item D'une gestion fine du cache CPU.
\end{itemize}

Dans ce projet, nous utilisons :
\begin{itemize}
    \item \texttt{cblas\_zdotu} : Produit scalaire complexe pour le calcul des coefficients.
    \item \texttt{cblas\_dnrm2} : Norme euclidienne pour la distance.
    \item \texttt{cblas\_daxpy} : Combinaison linéaire de vecteurs.
\end{itemize}

\chapter{Implémentation et Résultats}
\section{Pipeline de Traitement}
\begin{enumerate}
    \item \textbf{Extraction} : Segmentation IA (U-2-Net / rembg) pour isoler la pièce.
    \item \textbf{Contour} : Extraction du contour externe unique.
    \item \textbf{Calcul} : TFD via OpenBLAS.
    \item \textbf{Comparaison} : Distance euclidienne sur les hautes fréquences.
\end{enumerate}

\section{Analyse de Performance}
Les benchmarks montrent une accélération significative :
\begin{itemize}
    \item Speedup moyen : $\times 10$ à $\times 50$.
    \item Scalabilité linéaire.
\end{itemize}

\section{Séparabilité des Classes}
L'analyse sur le dataset de casting (OK vs DEF) montre :
\begin{itemize}
    \item Une séparation nette sur les hautes fréquences.
    \item L'importance cruciale de la qualité d'extraction du contour.
\end{itemize}

\chapter{Conclusion}
L'intégration d'OpenBLAS permet non seulement d'accélérer le calcul, mais ouvre la voie au traitement temps réel sur des lignes de production à haute cadence. L'approche hybride (Segmentation IA + Descripteurs mathématiques) offre un compromis optimal entre robustesse et explicabilité.

\end{document}
